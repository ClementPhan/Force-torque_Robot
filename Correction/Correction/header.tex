\usepackage[dvipsnames]{xcolor}
\usepackage{amsfonts}
\usepackage{amsmath}
\usepackage{amssymb}
\usepackage{amsthm}
\usepackage[french]{babel}
\usepackage{beramono}
\usepackage{changepage}
\usepackage{color}
\usepackage{mathdots}
\usepackage[utf8]{inputenc}
\usepackage{enumitem}
\usepackage{fancyhdr}
\usepackage{float}
\usepackage{framed}
\usepackage[T1]{fontenc}
\usepackage[margin=1in]{geometry}
\usepackage{listings}
\usepackage{mathrsfs}
\usepackage{tikz, tkz-tab}
\usepackage[european resistor]{circuitikz}
\usepackage{titling}
\usepackage{tabularx}
\usepackage{graphicx}
\usepackage{textcomp}
\usepackage{gensymb} %en tant que le package 'was'
\usepackage{physics}
\usepackage{marvosym}
\usepackage{standalone}
\usepackage{preview}
\usepackage{mathtools}
\usepackage[hidelinks]{hyperref}
\hypersetup{allcolors=black}


\usetikzlibrary{decorations.pathmorphing,patterns}

\newtheoremstyle{dotless}{}{}{\itshape}{}{\bfseries}{}{ }{}
\theoremstyle{dotless}

\newtheorem{defs}{D\'efinition}[subsection]
\newenvironment{defi}{\definecolor{shadecolor}{RGB}{255,236,217}\begin{shaded}\begin{defs}\ \\}{\end{defs}\end{shaded}}

\newtheorem{pro}{Propri\'et\'e}[subsection]
\newenvironment{prop}{\definecolor{shadecolor}{RGB}{230,230,255}\begin{shaded}\begin{pro}\ \\}{\end{pro}\end{shaded}}

\newtheorem*{theos}{Th\'eor\`eme}
\newenvironment{theo}[1]{\definecolor{shadecolor}{RGB}{230,230,255}\begin{shaded}\begin{theos}\textbf{\emph{#1}}\ \\}{\end{theos}\end{shaded}}

\newtheorem{cor}{Corollaire}[subsection]
\newenvironment{coro}{\definecolor{shadecolor}{RGB}{245,250,255}\begin{shaded}\begin{cor}\ \\}{\end{cor}\end{shaded}}

\setlength{\droptitle}{-1in}
%\predate{}
%\postdate{}
%\title{\textbf{\titre}\\\soustitre\vspace{-.18in}}

\pagestyle{fancy}
\makeatletter
\lhead{\title}
\rhead{\@author}
\makeatother

\newenvironment{preuve}{\begin{framed}\begin{proof}[\unskip\nopunct]}{\end{proof}\end{framed}}
\newenvironment{liste}{\begin{itemize}[leftmargin=*,noitemsep, topsep=0pt]}{\end{itemize}}
\newenvironment{tab}{\begin{adjustwidth}{.5cm}{}}{\end{adjustwidth}}

\newcommand{\lbracket}{[\![}
\newcommand{\rbracket}{]\!]}
\newcommand{\fonction}[5]{\begin{aligned}[t]#1\colon#2&\longrightarrow#3 \\#4&\longmapsto#5\end{aligned}}
\newcommand{\systeme}[1]{\left\{\begin{aligned}#1\end{aligned}\right.}
\newcommand{\cercle}[1]{\textcircled{\scriptsize{#1}}}
\newcommand{\Sup}{\mathop{\text{Sup}}}
\newcommand{\Inf}{\mathop{\text{Inf}}}
\newcommand{\lf}[1]{\left(#1\right)}
%\newcommand{\C}{\mathbb{C}}
\newcommand{\R}{\mathbb{R}}
\newcommand{\K}{\mathbb{K}}
\newcommand{\Q}{\mathbb{Q}}
\newcommand{\N}{\mathbb{N}}
\newcommand{\Z}{\mathbb{Z}}
\newcommand{\I}{\mathcal{I}}
\newcommand{\F}{\mathcal{F}}
\newcommand{\E}{\mathcal{E}}
%\newcommand{\G}{\mathcal{G}}
\newcommand{\et}{\text{ et }}
\newcommand{\ou}{\text{ ou }}
\newcommand{\xou}{\ \fbox{\text{ou}}\ }
\newcommand{\vdashv}{\mathrel{\text{\ooalign{$\vdash$\cr$\dashv$\cr}}}}
\newcommand{\card}{\text{card}}
\newcommand{\restr}[2]{{% we make the whole thing an ordinary symbol
		\left.\kern-\nulldelimiterspace % automatically resize the bar with \right
		#1 % the function
		\vphantom{\big|} % pretend it's a little taller at normal size
		\right|_{#2} % this is the delimiter
	}}
\newcommand{\floor}[1]{\lfloor #1 \rfloor}
\newcommand{\ceil}[1]{\lceil #1 \rceil}
\newcommand{\Sev}{\mathrm{S}_\mathrm{EV}}
\newcommand{\vect}{\mathrm{Vect}}
\newcommand{\Kev}{\K_\mathrm{EV}}

%%
%% Julia definition (c) 2014 Jubobs
%%
\lstdefinelanguage{Julia}%
{
	morekeywords={abstract,break,case,catch,const,continue,do,else,elseif,%
		end,export,false,for,function,immutable,import,importall,if,in,%
		macro,module,otherwise,quote,return,switch,true,try,type,typealias,%
		using,while},%
	sensitive=false,%
	alsoother={$},%
	morecomment=[l]\#,%
	morecomment=[n]{\#=}{=\#},%
	morestring=[s]{"}{"},%
	morestring=[m]{'}{'},%
}[keywords,comments,strings]%

%%
%% Laplacien et D'Alembertien
%%
\newcommand*\Laplace{\mathop{}\!\mathbin\bigtriangleup}
\newcommand*\DAlambert{\mathop{}\!\mathbin\Box}
